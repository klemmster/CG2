\documentclass[a4paper,oneside,12pt]{scrartcl}

\usepackage[latin1]{inputenc}
\usepackage{amsmath,amssymb,graphicx,latexsym,stmaryrd,dsfont,amsthm}
\usepackage[T1]{fontenc}
\usepackage{ngerman,url,booktabs,scrpage2}
\usepackage{color}
\usepackage{algpseudocode}

\newcommand{\heading}[1]{\item{\textbf{#1}}$ $\\}
\newcommand{\formel}[1]{\begin{center}#1\end{center}}
\newcommand{\eqs}[1]{
$\begin{array}{rcll}
$ #1 $
\end{array}$}
\newcommand{\image}[2]{
\begin{center}
\includegraphics[scale = #1]{#2}
\end{center}
}

\newcounter{counter}
\stepcounter{counter}

\definecolor{gray}{gray}{.5}
\newcommand{\comment}[1]{\textcolor{gray}{\textit{// #1}}}
\newcommand{\task}[1]{
\section*{Aufgabe #1}
\stepcounter{counter}
}
\newcommand{\subtask}[1]{\subsection*{#1}}
\newcommand{\subsubtask}[1]{#1$ $\\}
\newcommand{\mathsum}[3]{\underset{#1}{\overset{#2}{\sum}}#3}
\newcommand{\isum}[1]{\mathsum{i=0}{n}{#1}}

\newcommand{\twovector}[2]{\left(\begin{array}{c} #1 \\ #2 \end{array}\right)}
\newcommand{\threevector}[3]{\left(\begin{array}{c} #1 \\ #2 \\ #3 \end{array}\right)}
\newcommand{\fourmatrix}[4]{\left(\begin{array}{cc} #1 & #2 \\ #3 & #4 \end{array}\right)}
\newcommand{\ninematrix}[9]{\left(\begin{array}{ccc} #1 & #2 & #3 \\ #4 & #5 & #6 \\ #7 & #8 & #9 \end{array}\right)}
\newcommand{\nvector}[2]{\left(\begin{array}{c} #1 \\ \vdots \\ #2 \end{array}\right)}
\newcommand{\nmatrix}[4]{\left(\begin{array}{ccc} #1 & \ldots & #2 \\ \vdots & \ddots & \vdots \\ #3 & \ldots & #4 \end{array}\right)}
\newcommand{\eq}{&=&}
\newcommand{\eqbreak}{\\[0.5cm]}
\newcommand{\eqsmbreak}{\\[0.35cm]}
\newcommand{\eqtbreak}{\\[0.12cm]}
\newcommand{\eqbreakp}{\\[0.8cm]}
\newcommand{\eqsmbreakp}{\\[0.65cm]}
\newcommand{\eqtbreakp}{\\[0.32cm]}
\newcommand{\rc}{r^c}
\newcommand{\vvec}{\vec v}
\newcommand{\pnt}[1]{\textbf{#1}}
\newcommand{\ppnt}{\pnt{p}}

\parindent 0pt

\begin{document}

\begin{center}
\normalsize Computer Graphics 2\\[0.3cm]
\huge \textbf{Hausaufgabe 2} \normalsize\\[0.8cm]
\begin{tabular}{rl}
\texttt{343635} & Richard Klemm\\
\texttt{319716} & Andreas Fender\\
\texttt{315744} & Christopher Sierigk\\
\end{tabular}
\end{center}

\stepcounter{counter}

\task{1}

Sei $t$ die Stelle, an der die Kurve geteilt und an dem der Punkt ausgewertet wird.\\
\\
Der Algorithmus von Casteljau funktioniert �ber die folgende Rekursionsformel:\\\\
\eqs{$
\pnt{b}_i^0(t) \eq \pnt{b}_i\\
\pnt{b}_i^r(t) \eq t\cdot \pnt{b}_{i+1}^{r-1}(t) + (1-t)\cdot \pnt{b}_i^{r-1}(t)
$}$ $\\\\
Das letzte Segment im Algorithmus von Casteljau entspricht dem Verbindungsvektor der Punkte der vorletzten Rekursionsstufe, d.h. der Differenz aus $\pnt{b}_1^{n-1}$ und $\pnt{b}_0^{n-1}$:
\formel{$\vec s = \pnt{b}_1^{n-1}(t) - \pnt{b}_0^{n-1}(t)$}
$ $\\
Die erste Ableitung ist gegeben durch:
\formel{$\pnt{p}'(t) = n(\pnt{b}_1^{n-1}(t) - \pnt{b}_0^{n-1}(t))$}
Somit gilt:
\formel{$\vec{s}(t) = n \pnt{p}'(t)$}
D.h. das letzte Segment und die Ableitung unterscheiden sich nur in dem Faktor $n$, womit diese linear abh�ngig sind. Somit ist das letzte Segment eine Tangente der Kurve im ausgewerteten Punkt.

\task{2}

Nein.\\
Gegeben seien zwei lineare Basen:\\\\
\eqs{$
A(x) \eq (1,x)\\
B(y) \eq (1,y)
$}$ $\\\\
Seien die Koeffizienten beliebig. Das Tensorprodukt ist dann:\\\\
\eqs{$
z(x,y) \eq \mathsum{i = 0}{1}{\mathsum{j = 0}{1}{c_{ij}\cdot A_i(x)B_j(y)}}\\
 			 \eq c_{00}\cdot 1 + c_{10}\cdot x + c_{01}\cdot y + c_{11}\cdot xy
$}$ $\\\\
Dies l�sst sich umstellen nach $c_{00}$:
\formel{$c_{10} x + c_{01}\cdot y + z + c_{11} xy = c_{00}$ $(1)$}
Ebenengleichung im 3-dimensionalen Raum ist jedoch:
\formel{$ax + by + cz = d$}
(1) l�sst sich wegen dem Summanden $c_{11} xy$ nicht in die Ebenengleichung �berf�hren. Daher ist die Fl�che im Allgemeinen (d.h. wenn $c_{11} \neq 0$) keine ebene Fl�che.

\task{3}

Das Kreuzprodukt zweier Vektoren liefert den Fl�cheninhalt des durch die Vektoren aufgespannten Parallelograms:
\formel{$A_p = \vec{v}_1 \times \vec{v}_2$}
Werden f�r diese Vektoren die Tangentialvektoren einer Fl�che am Punkt $(x,y)$ eingesetzt, erh�lt man die Fl�chenverzerrung am selbigen Punkt. Um nun den Gesamtfl�cheninhalt $A$ zu erhalten, muss innerhalb der Grenzen der Fl�che �ber die Fl�chenverzerrung integriert werden:
\formel{$A = \int_a^b\int_c^d \frac{\partial p(x,y)}{\partial x} \times \frac{\partial p(x,y)}{\partial y}$ $dx$ $dy$}

\task{4}

\textbf{a)}

Betrachte die folgende Parametrisierung:
\formel{$\pnt{p}(u,v) = \threevector{u}{v}{uv}$}
Die partiellen Ableitungen, bzw die Tangenten sind:\\\\
\eqs{$\frac{\partial p}{\partial u} = \threevector{1}{0}{v}$}
\eqs{$\frac{\partial p}{\partial v} = \threevector{0}{1}{u}$}$ $\\\\
$\pnt{p}$ ist differenzierbar und die Tangenten sind linear unabh�ngig. Damit ist $\pnt{p}$ regul�r. Allerdings sind die Tangenten nur in $\pnt{p}(0,0)$ orthonormal. F�r alle anderen Punkte gilt:\\\\
\eqs{$|\frac{\partial p}{\partial u}| \neq 1$}
\eqs{$|\frac{\partial p}{\partial v}| \neq 1$}$ $\\\\
Es gibt auch keine Umparametrisierung, die die Tangenten in allen Punkten orthonormal werden l�sst.\\

\textbf{b)}

Gegeben sei die folgende regul�re \pnt{q}(u,v):
\formel{$\pnt{q}(u,v) = \threevector{0.5u}{v}{1}$}
Die Tangenten sind:
\begin{center}
\eqs{$\frac{\partial q}{\partial u} \eq \threevector{0.5}{0}{0}$}
\eqs{$\frac{\partial q}{\partial v} \eq \threevector{0}{1}{0}$}
\end{center}
Die Tangenten sind offensichtlich linear unabh�ngig, aber nicht orthonormal. Eine entsprechende Umparametrisierung sieht wie folgt aus:
\formel{$\pnt{q}(u,v) = \threevector{u}{v}{1}$}
Die Tangenten sind:
\begin{center}
\eqs{$\frac{\partial q}{\partial u} \eq \threevector{1}{0}{0}$}
\eqs{$\frac{\partial q}{\partial v} \eq \threevector{0}{1}{0}$}
\end{center}
Die Parametrisierung erzeugt die gleiche Fl�che, allerdings sind die Tangenten orthonormal.

\task{5}




\end{document}
