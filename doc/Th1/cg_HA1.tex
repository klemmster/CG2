\documentclass[a4paper,oneside,12pt]{scrartcl}

\usepackage[latin1]{inputenc}
\usepackage{amsmath,amssymb,graphicx,latexsym,stmaryrd,dsfont,amsthm}
\usepackage[T1]{fontenc}
\usepackage{ngerman,url,booktabs,scrpage2}
\usepackage{color}
\usepackage{algpseudocode}

\newcommand{\heading}[1]{\item{\textbf{#1}}$ $\\}
\newcommand{\formel}[1]{\begin{center}#1\end{center}}
\newcommand{\eqs}[1]{
$\begin{array}{rcll}
$ #1 $
\end{array}$}
\newcommand{\image}[2]{
\begin{center}
\includegraphics[scale = #1]{#2}
\end{center}
}

\newcounter{counter}
\stepcounter{counter}

\definecolor{gray}{gray}{.5}
\newcommand{\comment}[1]{\textcolor{gray}{\textit{// #1}}}
\newcommand{\task}[1]{
\section*{Aufgabe #1}
\stepcounter{counter}
}
\newcommand{\subtask}[1]{\subsection*{#1}}
\newcommand{\subsubtask}[1]{#1$ $\\}
\newcommand{\mathsum}[3]{\underset{#1}{\overset{#2}{\sum}}#3}
\newcommand{\isum}[1]{\mathsum{i=0}{n}{#1}}

\newcommand{\twovector}[2]{\left(\begin{array}{c} #1 \\ #2 \end{array}\right)}
\newcommand{\threevector}[3]{\left(\begin{array}{c} #1 \\ #2 \\ #3 \end{array}\right)}
\newcommand{\fourmatrix}[4]{\left(\begin{array}{cc} #1 & #2 \\ #3 & #4 \end{array}\right)}
\newcommand{\ninematrix}[9]{\left(\begin{array}{ccc} #1 & #2 & #3 \\ #4 & #5 & #6 \\ #7 & #8 & #9 \end{array}\right)}
\newcommand{\nvector}[2]{\left(\begin{array}{c} #1 \\ \vdots \\ #2 \end{array}\right)}
\newcommand{\nmatrix}[4]{\left(\begin{array}{ccc} #1 & \ldots & #2 \\ \vdots & \ddots & \vdots \\ #3 & \ldots & #4 \end{array}\right)}
\newcommand{\eq}{&=&}
\newcommand{\eqbreak}{\\[0.5cm]}
\newcommand{\eqsmbreak}{\\[0.35cm]}
\newcommand{\eqtbreak}{\\[0.12cm]}
\newcommand{\eqbreakp}{\\[0.8cm]}
\newcommand{\eqsmbreakp}{\\[0.65cm]}
\newcommand{\eqtbreakp}{\\[0.32cm]}
\newcommand{\rc}{r^c}
\newcommand{\vvec}{\vec v}

\parindent 0pt

\begin{document}

\stepcounter{counter}

\task{1}

Um den Median in linearer Zeit zu bestimmen, kann der Median of Medians Algorithmus verwendet werden. Im Folgenden ist der Algorithmus in Form von Pseudocode n�her erl�utert.
\\
\begin{algorithmic}
\Function{select}{L, k}
\If {$L\ enth"alt\ 10\ oder\ weniger\ Elemente$}
\State sort(L)
\State \Return Element an $k$ter Stelle
\EndIf

\\
\\ Teile L in Subsets $S[i]$ mit jeweils 5 Elementen auf \Comment{Es gibt n/5 Subsets}
\\
\For {$i = 1 \to n/5$} 
\State $X\ add\ select(S[i], 3)$ \Comment{Median von den Subsets}
\EndFor
\\
\State $M = select(X[i], n/10)$
\\
\\ Teile L auf in L1 < M, L2 = M, L3 > M
\If {$k \leq |L1|$}
\State \Return {$select(L1, k)$}
\ElsIf{$k > |L1| + |L2|$}
\State $select(L3, k - |L1| - |L2|)$
\Else
\State $return\ M$
\EndIf
\EndFunction
\end{algorithmic}
$ $\\
\textbf{Aussage:} Der o.g. Algorithmus liefert den Median in $O(n)$.\\
\\
\textsc{Beweis:} Wir werfen entweder $L3$ (Werte gr��er als $M$) oder $L1$ (Werte kleiner als $M$) weg. Angenommen wir werfen $L3$ weg. Unter den $n/5$ Werten von $X$ befinden sich $n/10$ gr"o�ere Werte als $M$. Wenn ein Wert aus $X$ gr"o�er als $M$ ist, dann sind genau 2 Werte aus $S[i]$ ebenfalls gr"o�er als der Wert aus $X$. Daraus folgt, dass $L3$ mindestens 3 Elemente in jeder der $n/10$ Gruppen von $S[i]$ hat, woraus sich mindestens $3n/10$ Elemente ergeben. Das gleiche gilt f"ur $L1$, wodurch sich h"ochstens $7n/10$ Elemente ergeben ($T(7n/10)$).

\task{2}

Eine Zelle der Kantenl�nge $s$ kann bei einem Punkt-Mindestabstand $\epsilon$ maximal $n_{max} = (\frac{s}{\epsilon} + 1)^2$ Punkte enthalten, da am meisten Punkte in die Zelle passen, wenn die Punkte Gitterf�rmig angeordnet sind:

{\centering\image{0.2}{images/gridSample.png}}

Ein Octree hat bei einer Tiefe $t$ maximal $n = 8^t$ Bl�tter, bzw. Punkte. Nach $t$ umgestellt ergibt sich f�r die maximale Tiefe bei $n_{max}$ Punkten:
\formel{$t_{max} = log_8(n_{max}) = 2\cdot log_8(\frac{s}{\epsilon} + 1)$}

\task{3}

Ein weiteres volumetrisches Primitiv w�re ein Prisma mit einem gleichseitigen Dreieck als Grundfl�che (Dreiecke k�nnen in 2D auf einfache Weise selbst�hnlich angeordnet werden). Eine Unterteilungsstrategie k�nnte das Ausgangsprimitiv folgenderma�en in acht Unterprimitive teilen:\\
\\
Die Spitze des Ausgangsprismas zeige in negative Y-Richtung. Alle Unterprimitive werden zuerst (d.h. als letzte Transformationsmatrix) um $0.5$ skaliert.\\
Folgende Translationen werden f�r die ersten vier Primitiven durchgef�hrt ($h$ sei die H�he des Ausgangsprimitivs):\\

$\threevector{0}{-\frac{h}{4}}{0}, \threevector{-\frac{h}{4}}{\frac{h}{4}}{0}, \threevector{\frac{h}{4}}{\frac{h}{4}}{0}, \threevector{0}{\frac{h}{4}}{0}$\\

Das Prisma mit der letztgenannten Translation wird vorher um 180� um die Y-Achse gedreht.\\
Folgende Translationen werden f�r die letzten vier Primitiven durchgef�hrt ($h$ sei die H�he des Ausgangsprimitivs, $l$ die L�nge in Z-Richtung):\\

$\threevector{0}{-\frac{h}{4}}{\frac{l}{2}}, \threevector{-\frac{h}{4}}{\frac{h}{4}}{\frac{l}{2}}, \threevector{\frac{h}{4}}{\frac{h}{4}}{\frac{l}{2}}, \threevector{0}{\frac{h}{4}}{\frac{l}{2}}$\\

Auch hier wird das Prisma mit der letztgenannten Translation vorher um 180� um die Y-Achse gedreht.

\task{4}

Sei $n$ der Grad.\\
$ $\\

{\centering\image{0.1}{images/bspline.jpg}}
Die Kontrollpunkte werden linear interpoliert. Die Basis dazu sieht folgenderma�en aus:
{\centering\image{0.3}{images/normalKnots.jpg}}

Ein B-Spline ist nicht stetig, wenn in der Basis mehr als $n$ Knots aufeinander liegen:
{\centering\image{0.3}{images/shiftKnots.jpg}}
In der Kurve entstehen unstetige Spr�nge:
{\centering\image{0.1}{images/bspline2.jpg}}

\task{5}

Seien:\\
$n$ die Anzahl der Kontrollpunkte\\
$p_i$ mit $i \in \{0,1,2 ... n\}$ die Kontrollpunkte\\
$ $\\
Eine affine Transformation l�sst sich in einen multiplikativen Teil (Matrix $M$) und einen additiven Teil (Vektor $\vec a$) aufteilen:

\formel{$T(\vvec) = M\cdot\vvec + \vec a$}

Affine Invarianz: Es ist egal, ob zuerst $T$ auf die Kontrollpunkte angewandt wird mit anschlie�ender Interpolation oder umgekehrt:
\formel{$T(\isum{p_i\cdot L_i^n(u)}) = \isum{(T(p_i\cdot L_i^n(u)))}$}
Setze gleich:\\
\eqs{$
T(\isum{p_i\cdot L_i^n(u)}) \eq \isum{(T(p_i\cdot L_i^n(u)))}\\
M(\isum{p_i\cdot L_i^n(u)}) + \vec a \eq \isum{(M(p_i\cdot L_i^n(u)) + \vec a)}\\
\isum{M(p_i\cdot L_i^n(u))} + \vec a \eq \isum{M(p_i\cdot L_i^n(u))} + \isum{\vec a\cdot L_i^n(u)}\\
\vec a \eq \vec a \cdot \isum{L_i^n(u)} & \Leftrightarrow \isum{L_i^n(u) = 1} \qed
$}
$ $\\\\\\
Demnach sind beide Seiten gleich, genau dann, wenn f�r die Basisfunktion gilt:
\formel{$L_i^n(u) = 1$ $\forall u$ //Partition der Eins}
\end{document}
