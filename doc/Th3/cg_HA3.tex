\documentclass[a4paper,oneside,12pt]{scrartcl}

\usepackage[latin1]{inputenc}
\usepackage{amsmath,amssymb,graphicx,latexsym,stmaryrd,dsfont,amsthm}
\usepackage[T1]{fontenc}
\usepackage{ngerman,url,booktabs,scrpage2}
\usepackage{color}
\usepackage{algpseudocode}

\newcommand{\heading}[1]{\item{\textbf{#1}}$ $\\}
\newcommand{\formel}[1]{\begin{center}#1\end{center}}
\newcommand{\eqs}[1]{
$\begin{array}{rcll}
$ #1 $
\end{array}$}
\newcommand{\image}[2]{
\begin{center}
\includegraphics[scale = #1]{#2}
\end{center}
}

\newcounter{counter}
\stepcounter{counter}

\definecolor{gray}{gray}{.5}
\newcommand{\comment}[1]{\textcolor{gray}{\textit{// #1}}}
\newcommand{\task}[1]{
\section*{Aufgabe #1}
\stepcounter{counter}
}
\newcommand{\subtask}[1]{\subsection*{#1}}
\newcommand{\subsubtask}[1]{#1$ $\\}
\newcommand{\mathsum}[3]{\underset{#1}{\overset{#2}{\sum}}#3}
\newcommand{\isum}[1]{\mathsum{i=0}{n}{#1}}

\newcommand{\twovector}[2]{\left(\begin{array}{c} #1 \\ #2 \end{array}\right)}
\newcommand{\threevector}[3]{\left(\begin{array}{c} #1 \\ #2 \\ #3 \end{array}\right)}
\newcommand{\fourmatrix}[4]{\left(\begin{array}{cc} #1 & #2 \\ #3 & #4 \end{array}\right)}
\newcommand{\ninematrix}[9]{\left(\begin{array}{ccc} #1 & #2 & #3 \\ #4 & #5 & #6 \\ #7 & #8 & #9 \end{array}\right)}
\newcommand{\nvector}[2]{\left(\begin{array}{c} #1 \\ \vdots \\ #2 \end{array}\right)}
\newcommand{\nmatrix}[4]{\left(\begin{array}{ccc} #1 & \ldots & #2 \\ \vdots & \ddots & \vdots \\ #3 & \ldots & #4 \end{array}\right)}
\newcommand{\eq}{&=&}
\newcommand{\eqbreak}{\\[0.5cm]}
\newcommand{\eqsmbreak}{\\[0.35cm]}
\newcommand{\eqtbreak}{\\[0.12cm]}
\newcommand{\eqbreakp}{\\[0.8cm]}
\newcommand{\eqsmbreakp}{\\[0.65cm]}
\newcommand{\eqtbreakp}{\\[0.32cm]}
\newcommand{\rc}{r^c}
\newcommand{\vvec}{\vec v}
\newcommand{\pnt}[1]{\textbf{#1}}
\newcommand{\ppnt}{\pnt{p}}

\parindent 0pt

\begin{document}

\begin{center}
\normalsize Computer Graphics 2\\[0.3cm]
\huge \textbf{Hausaufgabe 2} \normalsize\\[0.8cm]
\begin{tabular}{rl}
\texttt{343635} & Richard Klemm\\
\texttt{319716} & Andreas Fender\\
\texttt{315744} & Christopher Sierigk\\
\end{tabular}
\end{center}

\stepcounter{counter}

\task{1}


\task{2}

Der Kreis ist ein Kegelschnitt einer zur X-Y-Ebene parallelen Ebene mit Kegeln die die Spitze bei (0,0,0) haben und Rotationssymmetrisch zur Z-Achse sind. Die implizite Darstellung sieht wie folgt aus:
\formel{$f(x,y) = x^2 + y^2 - r^2$}
Wobei $\{(x,y) | f(x,y) = 0\}$ den Kreis bilden. $r$ ist sowohl die H�he der Ebene als auch der Radius des entstehenden Kreises. 

Seien $f_1(x,y)$ und $f_2(x,y)$ zwei Kreise mit den Radien $r_1$ und $r_2$. Die algebraische Summe sieht dann wie folgt aus:
\formel{$f_1(x,y) + f_2(x,y) = 2x^2 + 2y^2 - r_1^2 - r_2^2$}
Der resultierende Kreis ist bei $f_1(x,y) + f_2(x,y) = 0$. Somit gilt:
\begin{center}
\eqs{$
2x^2 + 2y^2 - r_1^2 - r_2^2 \eq 0 & | \cdot \frac12\eqbreak
x^2 + y^2 - \frac{r_1^2 + r_2^2}2 \eq 0
$}
\end{center}
Demnach entsteht ein Kreis, dessen quadratischer Radius der Durchschnitt aus $r_1^2$ und $r_2^2$ ist, d.h.
\formel{$R^2 = \frac{r_1^2 + r_2^2}2$}
Auf beiden Seiten kann $\pi$ multipliziert werden:
\begin{center}
\eqs{$
R^2 \eq \frac{r_1^2 + r_2^2}2 & | \cdot \pi\eqbreak
\pi\cdot R^2 \eq \pi \cdot \frac{r_1^2 + r_2^2}2 \eqbreak
R^2\pi \eq \frac{r_1^2\pi + r_2^2\pi}2
$}
\end{center}
Somit kann auch gesagt werden, dass der resultierende Kreis den Druchschnittsfl�cheninhalt der beiden Kreise hat.

\task{4}

Ja.\\
Marching Cubes fragt eine endliche Menge von Sampling Punkten ab. Scharfe Kanten (Features) werden nicht erkannt. Auch bei einer Erh�hung der Sampling Punkte werden Features nicht erkannt, da diese sich innerhalb eines Cubes befinden.

\end{document}
