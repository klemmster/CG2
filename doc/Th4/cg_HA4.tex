\documentclass[a4paper,oneside,12pt]{scrartcl}

\usepackage[latin1]{inputenc}
\usepackage{amsmath,amssymb,graphicx,latexsym,stmaryrd,dsfont,amsthm}
\usepackage[T1]{fontenc}
\usepackage{ngerman,url,booktabs,scrpage2}
\usepackage{color}
\usepackage{algpseudocode}

\newcommand{\heading}[1]{\item{\textbf{#1}}$ $\\}
\newcommand{\formel}[1]{\begin{center}#1\end{center}}
\newcommand{\eqs}[1]{
$\begin{array}{rcll}
$ #1 $
\end{array}$}
\newcommand{\image}[2]{
\begin{center}
\includegraphics[scale = #1]{#2}
\end{center}
}

\newcounter{counter}
\stepcounter{counter}

\definecolor{gray}{gray}{.5}
\newcommand{\comment}[1]{\textcolor{gray}{\textit{// #1}}}
\newcommand{\task}[1]{
\section*{Aufgabe #1}
\stepcounter{counter}
}
\newcommand{\subtask}[1]{\subsection*{#1}}
\newcommand{\subsubtask}[1]{#1$ $\\}
\newcommand{\mathsum}[3]{\underset{#1}{\overset{#2}{\sum}}#3}
\newcommand{\isum}[1]{\mathsum{i=0}{n}{#1}}

\newcommand{\twovector}[2]{\left(\begin{array}{c} #1 \\ #2 \end{array}\right)}
\newcommand{\threevector}[3]{\left(\begin{array}{c} #1 \\ #2 \\ #3 \end{array}\right)}
\newcommand{\fourmatrix}[4]{\left(\begin{array}{cc} #1 & #2 \\ #3 & #4 \end{array}\right)}
\newcommand{\ninematrix}[9]{\left(\begin{array}{ccc} #1 & #2 & #3 \\ #4 & #5 & #6 \\ #7 & #8 & #9 \end{array}\right)}
\newcommand{\nvector}[2]{\left(\begin{array}{c} #1 \\ \vdots \\ #2 \end{array}\right)}
\newcommand{\nmatrix}[4]{\left(\begin{array}{ccc} #1 & \ldots & #2 \\ \vdots & \ddots & \vdots \\ #3 & \ldots & #4 \end{array}\right)}
\newcommand{\eq}{&=&}
\newcommand{\eqbreak}{\\[0.5cm]}
\newcommand{\eqsmbreak}{\\[0.35cm]}
\newcommand{\eqtbreak}{\\[0.12cm]}
\newcommand{\eqbreakp}{\\[0.8cm]}
\newcommand{\eqsmbreakp}{\\[0.65cm]}
\newcommand{\eqtbreakp}{\\[0.32cm]}
\newcommand{\rc}{r^c}
\newcommand{\vvec}{\vec v}
\newcommand{\pnt}[1]{\textbf{#1}}
\newcommand{\ppnt}{\pnt{p}}

\parindent 0pt

\begin{document}

\begin{center}
\normalsize Computer Graphics 2\\[0.3cm]
\huge \textbf{Hausaufgabe 3} \normalsize\\[0.8cm]
\begin{tabular}{rl}
\texttt{343635} & Richard Klemm\\
\texttt{319716} & Andreas Fender\\
\texttt{315744} & Christopher Sierigk\\
\end{tabular}
\end{center}

\stepcounter{counter}

\task{1}

Ein Polyeder ist ein Polygonnetz, dass geschlossen ist. Diese Eigenschaft ist wichtig, da an jedem Punkt klar definiert sein muss, ob dieser innerhalb oder au�erhalb der Fl�che ist.

\task{2}


\task{3}

\textbf{a) Kante teilen}\\

\image{0.34}{res/EdgeSplit.png}

\textbf{b) Kante entfernen}\\



\task{5}

Dreiecksgitter k�nnen Sechseckgitter enthalten. Aus einem Seckseckgitter l�sst sich immer ein Dreiecksgitter gewinnen, indem z.B. f�r jedes Sechseck ein Vertex im Zentrum eingef�gt wird und Edges zu allen Vertices des Sechsecks erstellt werden:

\image{0.26}{res/hexTri.png}

Das entstandene Dreiecksgitter kann nun verfeinert werden, indem im Mittelpunkt jedes Dreiecks ein Vertex eingef�gt und dieser mit den jeweiligen Dreiecksvertices verbunden wird. Das entstehende feinere Dreiecksnetz enth�lt auch eine verfeinerte Version des Ausgangssechsecknetzes:

\image{0.26}{res/hexRefine.png}

Die ausgegrauten Vertices und Edges m�ssen demnach entfernt werden.\\
Die folgende Abbildung zeigt, dass die entstandene Verfeinerung wiederholend ist:

\image{0.26}{res/hexPatch.png}


\end{document}
